% Options for packages loaded elsewhere
\PassOptionsToPackage{unicode}{hyperref}
\PassOptionsToPackage{hyphens}{url}
%
\documentclass[
]{article}
\usepackage{amsmath,amssymb}
\usepackage{lmodern}
\usepackage{ifxetex,ifluatex}
\ifnum 0\ifxetex 1\fi\ifluatex 1\fi=0 % if pdftex
  \usepackage[T1]{fontenc}
  \usepackage[utf8]{inputenc}
  \usepackage{textcomp} % provide euro and other symbols
\else % if luatex or xetex
  \usepackage{unicode-math}
  \defaultfontfeatures{Scale=MatchLowercase}
  \defaultfontfeatures[\rmfamily]{Ligatures=TeX,Scale=1}
\fi
% Use upquote if available, for straight quotes in verbatim environments
\IfFileExists{upquote.sty}{\usepackage{upquote}}{}
\IfFileExists{microtype.sty}{% use microtype if available
  \usepackage[]{microtype}
  \UseMicrotypeSet[protrusion]{basicmath} % disable protrusion for tt fonts
}{}
\makeatletter
\@ifundefined{KOMAClassName}{% if non-KOMA class
  \IfFileExists{parskip.sty}{%
    \usepackage{parskip}
  }{% else
    \setlength{\parindent}{0pt}
    \setlength{\parskip}{6pt plus 2pt minus 1pt}}
}{% if KOMA class
  \KOMAoptions{parskip=half}}
\makeatother
\usepackage{xcolor}
\IfFileExists{xurl.sty}{\usepackage{xurl}}{} % add URL line breaks if available
\IfFileExists{bookmark.sty}{\usepackage{bookmark}}{\usepackage{hyperref}}
\hypersetup{
  pdftitle={Assignment 3. Working with Data Frame. Base R Style},
  hidelinks,
  pdfcreator={LaTeX via pandoc}}
\urlstyle{same} % disable monospaced font for URLs
\usepackage[margin=1in]{geometry}
\usepackage{color}
\usepackage{fancyvrb}
\newcommand{\VerbBar}{|}
\newcommand{\VERB}{\Verb[commandchars=\\\{\}]}
\DefineVerbatimEnvironment{Highlighting}{Verbatim}{commandchars=\\\{\}}
% Add ',fontsize=\small' for more characters per line
\usepackage{framed}
\definecolor{shadecolor}{RGB}{248,248,248}
\newenvironment{Shaded}{\begin{snugshade}}{\end{snugshade}}
\newcommand{\AlertTok}[1]{\textcolor[rgb]{0.94,0.16,0.16}{#1}}
\newcommand{\AnnotationTok}[1]{\textcolor[rgb]{0.56,0.35,0.01}{\textbf{\textit{#1}}}}
\newcommand{\AttributeTok}[1]{\textcolor[rgb]{0.77,0.63,0.00}{#1}}
\newcommand{\BaseNTok}[1]{\textcolor[rgb]{0.00,0.00,0.81}{#1}}
\newcommand{\BuiltInTok}[1]{#1}
\newcommand{\CharTok}[1]{\textcolor[rgb]{0.31,0.60,0.02}{#1}}
\newcommand{\CommentTok}[1]{\textcolor[rgb]{0.56,0.35,0.01}{\textit{#1}}}
\newcommand{\CommentVarTok}[1]{\textcolor[rgb]{0.56,0.35,0.01}{\textbf{\textit{#1}}}}
\newcommand{\ConstantTok}[1]{\textcolor[rgb]{0.00,0.00,0.00}{#1}}
\newcommand{\ControlFlowTok}[1]{\textcolor[rgb]{0.13,0.29,0.53}{\textbf{#1}}}
\newcommand{\DataTypeTok}[1]{\textcolor[rgb]{0.13,0.29,0.53}{#1}}
\newcommand{\DecValTok}[1]{\textcolor[rgb]{0.00,0.00,0.81}{#1}}
\newcommand{\DocumentationTok}[1]{\textcolor[rgb]{0.56,0.35,0.01}{\textbf{\textit{#1}}}}
\newcommand{\ErrorTok}[1]{\textcolor[rgb]{0.64,0.00,0.00}{\textbf{#1}}}
\newcommand{\ExtensionTok}[1]{#1}
\newcommand{\FloatTok}[1]{\textcolor[rgb]{0.00,0.00,0.81}{#1}}
\newcommand{\FunctionTok}[1]{\textcolor[rgb]{0.00,0.00,0.00}{#1}}
\newcommand{\ImportTok}[1]{#1}
\newcommand{\InformationTok}[1]{\textcolor[rgb]{0.56,0.35,0.01}{\textbf{\textit{#1}}}}
\newcommand{\KeywordTok}[1]{\textcolor[rgb]{0.13,0.29,0.53}{\textbf{#1}}}
\newcommand{\NormalTok}[1]{#1}
\newcommand{\OperatorTok}[1]{\textcolor[rgb]{0.81,0.36,0.00}{\textbf{#1}}}
\newcommand{\OtherTok}[1]{\textcolor[rgb]{0.56,0.35,0.01}{#1}}
\newcommand{\PreprocessorTok}[1]{\textcolor[rgb]{0.56,0.35,0.01}{\textit{#1}}}
\newcommand{\RegionMarkerTok}[1]{#1}
\newcommand{\SpecialCharTok}[1]{\textcolor[rgb]{0.00,0.00,0.00}{#1}}
\newcommand{\SpecialStringTok}[1]{\textcolor[rgb]{0.31,0.60,0.02}{#1}}
\newcommand{\StringTok}[1]{\textcolor[rgb]{0.31,0.60,0.02}{#1}}
\newcommand{\VariableTok}[1]{\textcolor[rgb]{0.00,0.00,0.00}{#1}}
\newcommand{\VerbatimStringTok}[1]{\textcolor[rgb]{0.31,0.60,0.02}{#1}}
\newcommand{\WarningTok}[1]{\textcolor[rgb]{0.56,0.35,0.01}{\textbf{\textit{#1}}}}
\usepackage{longtable,booktabs,array}
\usepackage{calc} % for calculating minipage widths
% Correct order of tables after \paragraph or \subparagraph
\usepackage{etoolbox}
\makeatletter
\patchcmd\longtable{\par}{\if@noskipsec\mbox{}\fi\par}{}{}
\makeatother
% Allow footnotes in longtable head/foot
\IfFileExists{footnotehyper.sty}{\usepackage{footnotehyper}}{\usepackage{footnote}}
\makesavenoteenv{longtable}
\usepackage{graphicx}
\makeatletter
\def\maxwidth{\ifdim\Gin@nat@width>\linewidth\linewidth\else\Gin@nat@width\fi}
\def\maxheight{\ifdim\Gin@nat@height>\textheight\textheight\else\Gin@nat@height\fi}
\makeatother
% Scale images if necessary, so that they will not overflow the page
% margins by default, and it is still possible to overwrite the defaults
% using explicit options in \includegraphics[width, height, ...]{}
\setkeys{Gin}{width=\maxwidth,height=\maxheight,keepaspectratio}
% Set default figure placement to htbp
\makeatletter
\def\fps@figure{htbp}
\makeatother
\setlength{\emergencystretch}{3em} % prevent overfull lines
\providecommand{\tightlist}{%
  \setlength{\itemsep}{0pt}\setlength{\parskip}{0pt}}
\setcounter{secnumdepth}{-\maxdimen} % remove section numbering
\ifluatex
  \usepackage{selnolig}  % disable illegal ligatures
\fi

\title{Assignment 3. Working with Data Frame. Base R Style}
\author{}
\date{\vspace{-2.5em}}

\begin{document}
\maketitle

\textbf{\emph{Note}:} \emph{This assignment practices working with Data
Frame using Base R.}

\textbf{\emph{How to do it?}}:

\begin{itemize}
\item
  Open the Rmarkdown file of this assignment
  (\href{fa2021_assignment3.Rmd}{link}) in Rstudio.
\item
  Right under each question, insert a code chunk (you can use the hotkey
  \texttt{Ctrl\ +\ Alt\ +\ I} to add a code chunk) and code the solution
  for the question.
\item
  \texttt{Knit} the rmarkdown file (hotkey: \texttt{Ctrl\ +\ Alt\ +\ K})
  to export an html.
\item
  Publish the html file to your Githiub Page.
\end{itemize}

\textbf{\emph{Submission}}: Submit the link on Github of the assignment
to Canvas under Assignment 3.

\begin{center}\rule{0.5\linewidth}{0.5pt}\end{center}

\hypertarget{problems}{%
\subsection{Problems}\label{problems}}

\hfill\break

\begin{enumerate}
\def\labelenumi{\arabic{enumi}.}
\tightlist
\item
  Create the following data frame
\end{enumerate}

\begin{longtable}[]{@{}lll@{}}
\toprule
Rank & Age & Name \\
\midrule
\endhead
0 & 28 & Tom \\
1 & 34 & Jack \\
2 & 29 & Steve \\
3 & 42 & Ricky \\
\bottomrule
\end{longtable}

\begin{Shaded}
\begin{Highlighting}[]
\NormalTok{df }\OtherTok{=} \FunctionTok{data.frame}\NormalTok{(}\AttributeTok{Rank =} \FunctionTok{c}\NormalTok{(}\StringTok{\textquotesingle{}0\textquotesingle{}}\NormalTok{,}\StringTok{\textquotesingle{}1\textquotesingle{}}\NormalTok{,}\StringTok{\textquotesingle{}2\textquotesingle{}}\NormalTok{,}\StringTok{\textquotesingle{}3\textquotesingle{}}\NormalTok{), }\AttributeTok{Age =} \FunctionTok{c}\NormalTok{(}\StringTok{\textquotesingle{}28\textquotesingle{}}\NormalTok{,}\StringTok{\textquotesingle{}34\textquotesingle{}}\NormalTok{,}\StringTok{\textquotesingle{}29\textquotesingle{}}\NormalTok{,}\StringTok{\textquotesingle{}42\textquotesingle{}}\NormalTok{), }\AttributeTok{Name =} \FunctionTok{c}\NormalTok{(}\StringTok{\textquotesingle{}Tom\textquotesingle{}}\NormalTok{,}\StringTok{\textquotesingle{}Jack\textquotesingle{}}\NormalTok{,}\StringTok{\textquotesingle{}Steve\textquotesingle{}}\NormalTok{,}\StringTok{\textquotesingle{}Ricky\textquotesingle{}}\NormalTok{))}
\end{Highlighting}
\end{Shaded}

\begin{enumerate}
\def\labelenumi{\arabic{enumi}.}
\setcounter{enumi}{1}
\tightlist
\item
  Use \texttt{read.csv} to import the Covid19 Vaccination data from WHO:
  \href{https://raw.githubusercontent.com/nytimes/covid-19-data/master/us-states.csv}{link}
\end{enumerate}

\begin{Shaded}
\begin{Highlighting}[]
\NormalTok{df }\OtherTok{\textless{}{-}} \FunctionTok{read.csv}\NormalTok{(}\StringTok{\textquotesingle{}https://raw.githubusercontent.com/nytimes/covid{-}19{-}data/master/us{-}states.csv\textquotesingle{}}\NormalTok{)}
\end{Highlighting}
\end{Shaded}

Show the names of the variables in the data

\begin{Shaded}
\begin{Highlighting}[]
\FunctionTok{names}\NormalTok{(df)}
\end{Highlighting}
\end{Shaded}

\begin{verbatim}
## [1] "date"   "state"  "fips"   "cases"  "deaths"
\end{verbatim}

\begin{enumerate}
\def\labelenumi{\arabic{enumi}.}
\setcounter{enumi}{2}
\tightlist
\item
  How many columns and rows the data have?
\end{enumerate}

\begin{Shaded}
\begin{Highlighting}[]
\FunctionTok{str}\NormalTok{(df)}
\end{Highlighting}
\end{Shaded}

\begin{verbatim}
## 'data.frame':    31254 obs. of  5 variables:
##  $ date  : chr  "2020-01-21" "2020-01-22" "2020-01-23" "2020-01-24" ...
##  $ state : chr  "Washington" "Washington" "Washington" "Illinois" ...
##  $ fips  : int  53 53 53 17 53 6 17 53 4 6 ...
##  $ cases : int  1 1 1 1 1 1 1 1 1 2 ...
##  $ deaths: int  0 0 0 0 0 0 0 0 0 0 ...
\end{verbatim}

\begin{enumerate}
\def\labelenumi{\arabic{enumi}.}
\setcounter{enumi}{3}
\tightlist
\item
  How many missing values are there? Show the missing values by columns.
  What variable has the most number of missing values?
\end{enumerate}

\begin{Shaded}
\begin{Highlighting}[]
\FunctionTok{sum}\NormalTok{(}\FunctionTok{is.na}\NormalTok{(df))}
\end{Highlighting}
\end{Shaded}

\begin{verbatim}
## [1] 0
\end{verbatim}

\begin{Shaded}
\begin{Highlighting}[]
\FunctionTok{colSums}\NormalTok{(}\FunctionTok{is.na}\NormalTok{(df))}
\end{Highlighting}
\end{Shaded}

\begin{verbatim}
##   date  state   fips  cases deaths 
##      0      0      0      0      0
\end{verbatim}

\begin{enumerate}
\def\labelenumi{\arabic{enumi}.}
\setcounter{enumi}{4}
\tightlist
\item
  What is the class of the \texttt{date} column. Change the
  \texttt{date} columns to \texttt{date} type using the \texttt{as.Date}
  function. Show the new class of the \texttt{date} column.
\end{enumerate}

\begin{Shaded}
\begin{Highlighting}[]
\FunctionTok{class}\NormalTok{(df}\SpecialCharTok{$}\NormalTok{date)}
\end{Highlighting}
\end{Shaded}

\begin{verbatim}
## [1] "character"
\end{verbatim}

\begin{Shaded}
\begin{Highlighting}[]
\NormalTok{df}\SpecialCharTok{$}\NormalTok{date }\OtherTok{=} \FunctionTok{as.Date}\NormalTok{(df}\SpecialCharTok{$}\NormalTok{date)}
\FunctionTok{class}\NormalTok{(df}\SpecialCharTok{$}\NormalTok{date)}
\end{Highlighting}
\end{Shaded}

\begin{verbatim}
## [1] "Date"
\end{verbatim}

\begin{enumerate}
\def\labelenumi{\arabic{enumi}.}
\setcounter{enumi}{5}
\tightlist
\item
  Capitalize the names of all the variables
\end{enumerate}

\begin{Shaded}
\begin{Highlighting}[]
\FunctionTok{names}\NormalTok{(df)[}\DecValTok{1}\NormalTok{] }\OtherTok{\textless{}{-}} \StringTok{\textquotesingle{}Date\textquotesingle{}}
\FunctionTok{names}\NormalTok{(df)[}\DecValTok{2}\NormalTok{] }\OtherTok{\textless{}{-}} \StringTok{\textquotesingle{}State\textquotesingle{}}
\FunctionTok{names}\NormalTok{(df)[}\DecValTok{3}\NormalTok{] }\OtherTok{\textless{}{-}} \StringTok{\textquotesingle{}Fips\textquotesingle{}}
\FunctionTok{names}\NormalTok{(df)[}\DecValTok{4}\NormalTok{] }\OtherTok{\textless{}{-}} \StringTok{\textquotesingle{}Cases\textquotesingle{}}
\FunctionTok{names}\NormalTok{(df)[}\DecValTok{5}\NormalTok{] }\OtherTok{\textless{}{-}} \StringTok{\textquotesingle{}Deaths\textquotesingle{}}
\FunctionTok{names}\NormalTok{(df)}
\end{Highlighting}
\end{Shaded}

\begin{verbatim}
## [1] "Date"   "State"  "Fips"   "Cases"  "Deaths"
\end{verbatim}

\begin{enumerate}
\def\labelenumi{\arabic{enumi}.}
\setcounter{enumi}{6}
\tightlist
\item
  Find the average number of cases per day. Find the maximum cases a
  day.
\end{enumerate}

\begin{Shaded}
\begin{Highlighting}[]
\FunctionTok{mean}\NormalTok{(df}\SpecialCharTok{$}\NormalTok{Cases, }\AttributeTok{na.rm=}\ConstantTok{TRUE}\NormalTok{)}
\end{Highlighting}
\end{Shaded}

\begin{verbatim}
## [1] 325915.5
\end{verbatim}

\begin{Shaded}
\begin{Highlighting}[]
\FunctionTok{max}\NormalTok{(df}\SpecialCharTok{$}\NormalTok{Cases, }\AttributeTok{na.rm=}\ConstantTok{TRUE}\NormalTok{)}
\end{Highlighting}
\end{Shaded}

\begin{verbatim}
## [1] 4669895
\end{verbatim}

\begin{enumerate}
\def\labelenumi{\arabic{enumi}.}
\setcounter{enumi}{7}
\tightlist
\item
  How many states are there in the data?
\end{enumerate}

\begin{Shaded}
\begin{Highlighting}[]
\FunctionTok{table}\NormalTok{(df}\SpecialCharTok{$}\NormalTok{State)}
\end{Highlighting}
\end{Shaded}

\begin{verbatim}
## 
##                  Alabama                   Alaska                  Arizona 
##                      558                      559                      605 
##                 Arkansas               California                 Colorado 
##                      560                      606                      566 
##              Connecticut                 Delaware     District of Columbia 
##                      563                      560                      564 
##                  Florida                  Georgia                     Guam 
##                      570                      569                      556 
##                   Hawaii                    Idaho                 Illinois 
##                      565                      558                      607 
##                  Indiana                     Iowa                   Kansas 
##                      565                      563                      564 
##                 Kentucky                Louisiana                    Maine 
##                      565                      562                      559 
##                 Maryland            Massachusetts                 Michigan 
##                      566                      599                      561 
##                Minnesota              Mississippi                 Missouri 
##                      565                      560                      564 
##                  Montana                 Nebraska                   Nevada 
##                      558                      583                      566 
##            New Hampshire               New Jersey               New Mexico 
##                      569                      567                      560 
##                 New York           North Carolina             North Dakota 
##                      570                      568                      560 
## Northern Mariana Islands                     Ohio                 Oklahoma 
##                      543                      562                      565 
##                   Oregon             Pennsylvania              Puerto Rico 
##                      572                      565                      558 
##             Rhode Island           South Carolina             South Dakota 
##                      570                      565                      561 
##                Tennessee                    Texas                     Utah 
##                      566                      588                      575 
##                  Vermont           Virgin Islands                 Virginia 
##                      564                      557                      564 
##               Washington            West Virginia                Wisconsin 
##                      610                      554                      595 
##                  Wyoming 
##                      560
\end{verbatim}

\begin{enumerate}
\def\labelenumi{\arabic{enumi}.}
\setcounter{enumi}{8}
\tightlist
\item
  Create a new variable \texttt{weekdays} to store the weekday for each
  rows.
\end{enumerate}

\begin{Shaded}
\begin{Highlighting}[]
\NormalTok{df}\SpecialCharTok{$}\NormalTok{Weekdays }\OtherTok{\textless{}{-}} \FunctionTok{weekdays}\NormalTok{(df}\SpecialCharTok{$}\NormalTok{Date)}
\end{Highlighting}
\end{Shaded}

\begin{enumerate}
\def\labelenumi{\arabic{enumi}.}
\setcounter{enumi}{9}
\tightlist
\item
  Create the categorical variable \texttt{death2} variable taking the
  values as follows
\end{enumerate}

\begin{itemize}
\tightlist
\item
  \texttt{has\_death} if there is a death that day
\item
  \texttt{no\_death} if there is no death that day
\end{itemize}

Find the frequency and relative frequency of \texttt{no\_death} and
\texttt{has\_death}.

\begin{Shaded}
\begin{Highlighting}[]
\NormalTok{df}\SpecialCharTok{$}\NormalTok{death2 }\OtherTok{\textless{}{-}} \FunctionTok{ifelse}\NormalTok{(df}\SpecialCharTok{$}\NormalTok{Deaths}\SpecialCharTok{==}\DecValTok{0}\NormalTok{,}\StringTok{\textquotesingle{}no\_death\textquotesingle{}}\NormalTok{,}\StringTok{\textquotesingle{}has\_death\textquotesingle{}}\NormalTok{)}
\FunctionTok{table}\NormalTok{(df}\SpecialCharTok{$}\NormalTok{Death2)}
\end{Highlighting}
\end{Shaded}

\begin{verbatim}
## < table of extent 0 >
\end{verbatim}

\begin{enumerate}
\def\labelenumi{\arabic{enumi}.}
\setcounter{enumi}{10}
\tightlist
\item
  Find the first quartile (Q1), second quartile (Q2) and and third
  quartile (Q3) of the variable \texttt{death}. (Hint: Use the
  \texttt{summary} function)
\end{enumerate}

\begin{Shaded}
\begin{Highlighting}[]
\FunctionTok{summary}\NormalTok{(df}\SpecialCharTok{$}\NormalTok{Deaths)}
\end{Highlighting}
\end{Shaded}

\begin{verbatim}
##    Min. 1st Qu.  Median    Mean 3rd Qu.    Max. 
##       0     365    2087    6204    7386   68244
\end{verbatim}

\begin{enumerate}
\def\labelenumi{\arabic{enumi}.}
\setcounter{enumi}{11}
\tightlist
\item
  Create the categorical variable \texttt{death3} variable taking the
  values as follows
\end{enumerate}

\begin{itemize}
\item
  \texttt{low\_death} if the number of deaths smaller than the 25
  percentile (Q1)
\item
  \texttt{mid\_death} if the number of deaths from Q1 to Q3
\item
  \texttt{high\_death} if the number of deaths greater than Q3
\end{itemize}

\begin{Shaded}
\begin{Highlighting}[]
\NormalTok{df}\SpecialCharTok{$}\NormalTok{death3 }\OtherTok{\textless{}{-}} \FunctionTok{cut}\NormalTok{(df}\SpecialCharTok{$}\NormalTok{Deaths, }\AttributeTok{breaks=}\FunctionTok{c}\NormalTok{(}\DecValTok{0}\NormalTok{,}\DecValTok{363}\NormalTok{,}\DecValTok{6183}\NormalTok{,}\ConstantTok{Inf}\NormalTok{), }\AttributeTok{labels=}\FunctionTok{c}\NormalTok{(}\StringTok{\textquotesingle{}low\_death\textquotesingle{}}\NormalTok{,}\StringTok{\textquotesingle{}mid\_death\textquotesingle{}}\NormalTok{,}\StringTok{\textquotesingle{}high\_death\textquotesingle{}}\NormalTok{))}
\end{Highlighting}
\end{Shaded}

\begin{enumerate}
\def\labelenumi{\arabic{enumi}.}
\setcounter{enumi}{12}
\tightlist
\item
  Find the average cases in Rhode Island in 2021
\end{enumerate}

\begin{Shaded}
\begin{Highlighting}[]
\NormalTok{df1 }\OtherTok{\textless{}{-}}\NormalTok{ df[df}\SpecialCharTok{$}\NormalTok{Date\_report}\SpecialCharTok{\textgreater{}=}\StringTok{\textquotesingle{}2021{-}01{-}01\textquotesingle{}}\NormalTok{,]}
\FunctionTok{by}\NormalTok{(df1}\SpecialCharTok{$}\NormalTok{New\_cases, df1}\SpecialCharTok{$}\NormalTok{WHO\_region, mean)}
\end{Highlighting}
\end{Shaded}

\begin{enumerate}
\def\labelenumi{\arabic{enumi}.}
\setcounter{enumi}{13}
\tightlist
\item
  Find the median cases by weekdays in Rhode Island in 2021
\end{enumerate}

\begin{Shaded}
\begin{Highlighting}[]
\FunctionTok{str}\NormalTok{(df}\SpecialCharTok{$}\NormalTok{State)}
\end{Highlighting}
\end{Shaded}

\begin{verbatim}
##  chr [1:31254] "Washington" "Washington" "Washington" "Illinois" ...
\end{verbatim}

\begin{enumerate}
\def\labelenumi{\arabic{enumi}.}
\setcounter{enumi}{14}
\tightlist
\item
  Compare the median cases in Rhode Island in June, July, August and
  September in 2021.
\end{enumerate}

\begin{Shaded}
\begin{Highlighting}[]
\FunctionTok{head}\NormalTok{(df}\SpecialCharTok{$}\NormalTok{State)}
\end{Highlighting}
\end{Shaded}

\begin{verbatim}
## [1] "Washington" "Washington" "Washington" "Illinois"   "Washington"
## [6] "California"
\end{verbatim}

\end{document}
